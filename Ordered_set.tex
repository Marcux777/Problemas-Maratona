\section{Ordered Set}
\subsection{List Removals}

\subsubsection*{Enunciado}
Você recebe uma lista composta por \( n \) inteiros. Sua tarefa é remover elementos da lista em posições dadas e relatar os elementos removidos.

\subsubsection*{Entrada}
A primeira linha de entrada contém um inteiro \( n \): o tamanho inicial da lista. Durante o processo, os elementos são numerados \( 1, 2, \dots, k \), onde \( k \) é o tamanho atual da lista.

A segunda linha contém \( n \) inteiros \( x_1, x_2, \dots, x_n \): o conteúdo da lista.

A última linha contém \( n \) inteiros \( p_1, p_2, \dots, p_n \): as posições dos elementos a serem removidos.

\subsubsection*{Saída}
Imprima os elementos na ordem em que foram removidos.

\subsubsection*{Restrições}
\[
1 \le n \le 2 \cdot 10^5
\]
\[
1 \le x_i \le 10^9
\]
\[
1 \le p_i \le n - i + 1
\]

\subsubsection*{Solução}
\begin{lstlisting}[language=C++]
void solve()
{
    int n; cin >> n;
    vi a(n);
    ordered_set<int> s;
    rep(i, 0, n){
        cin >> a[i];
        s.insert(i);
    }
    rep(i, 0, n){
        int p; cin >> p;
        int pos = *s.find_by_order(p-1);
        cout << a[pos] << " ";
        s.erase(pos);
    }
}
\end{lstlisting}


\subsection{Poblema 2}

\subsubsection*{Enunciado}
Você recebe um array de \( n \) inteiros. Sua tarefa é calcular a mediana de cada janela de \( k \) elementos, da esquerda para a direita.

A mediana é o elemento do meio quando os elementos estão ordenados. Se o número de elementos for par, existem duas medianas possíveis e assumimos que a mediana é a menor delas.

\subsubsection*{Entrada}
A primeira linha contém dois inteiros \( n \) e \( k \): o número de elementos e o tamanho da janela.

Em seguida, há \( n \) inteiros \( x_1, x_2, \dots, x_n \): o conteúdo do array.

\subsubsection*{Saída}
Imprima \( n-k+1 \) valores: as medianas.

\subsubsection*{Restrições}
\[
1 \le k \le n \le 2 \cdot 10^5
\]
\[
1 \le x_i \le 10^9, \quad \text{para } i = 1, 2, \dots, n.
\]


\subsubsection*{Solução}
\begin{lstlisting}[language=C++]
void solve()
{
    int n, k; cin >> n >> k;
    vi a(n);
    ordered_set<pii> s;
    rep(i, 0, n){
        cin >> a[i];
        if(i < k) // Insere apenas os primeiros k elementos.
            s.insert({a[i], i});
    }
    // a mediana será o elemento 
    // na posição (k-1)/2 (0-indexed)
    auto med = s.find_by_order((k-1)/2);
    cout << med->f << ' ';
    rep(i, k, n){
        // Insere o novo elemento com seu índice.
        s.insert({a[i], i});
        // Remove o elemento que saiu da janela.
        s.erase(s.find({a[i-k], i-k}));
        // Busca a mediana da janela atual.
        med = s.find_by_order((k-1)/2);
        cout << med->f << ' ';
    }
    cout << endl;
}
    
\end{lstlisting}
