\section{Componentes Fortemente Conexos}
\subsection{Dominós}

\subsubsection*{Enunciado}
Dominós são muito divertidos. Crianças gostam de posicionar as peças de dominó em pé, formando longas linhas. Quando um dominó cai, ele derruba o próximo, que por sua vez derruba o seguinte, e assim por diante, até o final da linha. Entretanto, às vezes uma peça de dominó não consegue derrubar a próxima; nesse caso, é preciso derrubá-la manualmente para que a reação em cadeia continue.

Sua tarefa é determinar, dado o arranjo de algumas peças de dominó, o número mínimo de peças que devem ser derrubadas manualmente para que todas as peças caiam.

\subsubsection*{Entrada}
A primeira linha da entrada contém um inteiro que especifica o número de casos de teste a seguir.  
Cada caso de teste começa com uma linha contendo dois inteiros, cada um não maior que 100.000. O primeiro inteiro \(n\) é o número de peças de dominó e o segundo inteiro \(m\) é o número de relações descritas em seguida. As peças de dominó são numeradas de \(1\) a \(n\).

Cada uma das \(m\) linhas seguintes contém dois inteiros \(x\) e \(y\) indicando que, se a peça \(x\) cair, ela fará a peça \(y\) cair também.

\subsubsection*{Saída}
Para cada caso de teste, imprima uma linha contendo um inteiro, o número mínimo de peças que devem ser derrubadas manualmente para que todas as peças de dominó caiam.

\subsubsection*{Exemplo de Entrada}
\begin{verbatim}
1
3 2
1 2
2 3
\end{verbatim}

\subsubsection*{Exemplo de Saída}
\begin{verbatim}
1
\end{verbatim}

\begin{lstlisting}
vi g[MAXN], rg[MAXN];
vector<bool> vis;
vi order;

void dfs1(int v)
{
    vis[v] = true;
    for (auto u : g[v])
        if (!vis[u])
            dfs1(u);
    order.pb(v);
}

void dfs2(int v, int cid, vi &comp_id)
{
    comp_id[v] = cid;
    for (auto u : rg[v])
        if (comp_id[u] == -1)
            dfs2(u, cid, comp_id);
}

int kosaraju(int n, vi &comp_id)
{
    vis.assign(n, false);
    order.clear();
    rep(i, 0, n)
        if (!vis[i])
            dfs1(i);

    reverse(all(order));
    comp_id.assign(n, -1);
    int cid = 0;
    for (auto v : order)
        if (comp_id[v] == -1) {
            dfs2(v, cid, comp_id);
            cid++;
        }
    return cid;
}

vvi condensation(int n, const vi &comp_id, int compCount)
{
    vvi condensed(compCount);
    vector<set<int>> temp(compCount);
    rep(u, 0, n)
        for (auto v : g[u])
            if (comp_id[u] != comp_id[v])
                temp[comp_id[u]].insert(comp_id[v]);

    rep(i, 0, compCount)
        for (auto v : temp[i])
            condensed[i].pb(v);

    return condensed;
}

void solve()
{
    int n, m; cin >> n >> m;
    rep(i, 0, n) {
        g[i].clear();
        rg[i].clear();
    }
    rep(i, 0, m){
        int x, y; cin >> x >> y;
        x--; y--;
        g[x].pb(y);
        rg[y].pb(x);
    }
    vi comp_id;
    int count = kosaraju(n, comp_id);
    auto condensed = condensation(n, comp_id, count);
    vi indegree(count, 0);
    rep (i, 0, count) {
        for (int v : condensed[i])
            indegree[v]++;
    }
    int ans = 0;
    rep (i, 0, count) {
        if (indegree[i] == 0)
            ans++;
    }
    cout << ans << endl;
}
\end{lstlisting}

\subsection{Come and Go}

\subsubsection*{Enunciado}
Em uma certa cidade há \(N\) interseções conectadas por ruas de mão única e de mão dupla. Trata-se de uma cidade moderna, em que várias ruas possuem túneis ou viadutos. Evidentemente, deve ser possível viajar entre quaisquer duas interseções. Mais precisamente, dados duas interseções \(V\) e \(W\), deve ser possível viajar de \(V\) para \(W\) e de \(W\) para \(V\).

Sua tarefa é escrever um programa que leia a descrição do sistema viário da cidade e determine se o requisito de conectividade está satisfeito ou não.

\subsubsection*{Entrada}
A entrada contém vários casos de teste. A primeira linha de um caso de teste contém dois inteiros \(N\) e \(M\), separados por um espaço, indicando o número de interseções (\(2 \le N \le 2000\)) e o número de ruas (\(2 \le M \le \frac{N(N-1)}{2}\)). As \(M\) linhas seguintes descrevem o sistema viário da cidade, onde cada linha descreve uma rua. A descrição de uma rua consiste de três inteiros \(V\), \(W\) e \(P\), separados por um espaço, onde \(V\) e \(W\) são identificadores distintos para as interseções (\(1 \le V, W \le N\), \(V \neq W\)) e \(P\) pode ser 1 ou 2; se \(P = 1\) a rua é de mão única, e o tráfego vai de \(V\) para \(W\); se \(P = 2\) a rua é de mão dupla e liga \(V\) e \(W\). Um par de interseções é conectado por, no máximo, uma rua.

O último caso de teste é seguido por uma linha que contém apenas dois números zero, separados por um espaço.

\subsubsection*{Saída}
Para cada caso de teste, seu programa deve imprimir uma única linha contendo um inteiro \(G\), onde \(G\) é igual a 1 se a condição de conectividade for satisfeita, e \(G\) é igual a 0 caso contrário.

\subsubsection*{Exemplo de Entrada}
\begin{verbatim}
4 5
1 2 1
1 3 2
2 4 1
3 4 1
4 1 2
3 2
1 2 2
1 3 2
3 2
1 2 2
1 3 1
4 2
1 2 2
3 4 2
0 0
\end{verbatim}

\subsubsection*{Exemplo de Saída}
\begin{verbatim}
1
1
0
0
\end{verbatim}


\begin{lstlisting}
    void dfs(int node, vector<bool>& visited, vector<vector<pair<int, int>>>& graph) {
    visited[node] = true;
    for (const auto& v : graph[node]) {
        int u = v.first;
        if (!visited[u]) {
            dfs(u, visited, graph);
        }
    }
}

bool isStronglyConnected(vector<vector<pair<int, int>>>& graph) {
    int n = graph.size();

    for (int i = 0; i < n; ++i) {
        vector<bool> visited(n, false);
        dfs(i, visited, graph);

        if (find(visited.begin(), visited.end(), false) != visited.end()) {
            return false;
        }
    }

    return true;
}

int main() {
    ios::sync_with_stdio(false);
    cin.tie(nullptr);
    cout.tie(nullptr);

    int n, m;
    while (cin >> n >> m && n && m) {
        vector<vector<pair<int, int>>> graph(n);

        for (int i = 0; i < m; ++i) {
            int x, y, z;
            cin >> x >> y >> z;
            graph[x - 1].emplace_back(y - 1, z);
            if (z == 2) {
                graph[y - 1].emplace_back(x - 1, z);
            }
        }

        cout << isStronglyConnected(graph) << endl;
    }

    return 0;
}
\end{lstlisting}