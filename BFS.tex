\section{BFS}
\subsection{Time To Live (TTL) - Problema de Alcance em Redes}

\subsubsection*{Enunciado}

Para evitar que mensagens de rede (pacotes) circulem indefinidamente dentro de uma rede, cada mensagem inclui um campo chamado \textbf{Time To Live (TTL)}. Esse campo contém o número de nós que podem retransmitir a mensagem antes que ela seja descartada.

Cada vez que um nó recebe uma mensagem, ele decrementa o valor do campo TTL em 1. Se a mensagem deve ser encaminhada e o campo TTL for zero, ela não é retransmitida.

Dado um conjunto de redes e consultas, determine quantos nós não são alcançáveis a partir de um nó inicial e um valor de TTL fornecido.

\subsubsection*{Entrada}

A entrada contém várias configurações de redes. Cada rede começa com um inteiro \( NC \), que especifica o número de conexões entre os nós. Um valor \( NC = 0 \) indica o fim da entrada.

Após \( NC \), seguem \( NC \) pares de inteiros representando conexões diretas entre nós. Nenhuma rede terá mais de 30 nós.

Após a descrição da rede, múltiplas consultas são fornecidas, compostas por dois inteiros: o nó inicial e o valor TTL. As consultas terminam com o par \( (0,0) \).

\subsubsection*{Saída}

Para cada consulta, exiba uma linha contendo:
\begin{itemize}
    \item O número do caso de teste.
    \item A quantidade de nós inalcançáveis.
    \item O nó inicial e o TTL fornecido.
\end{itemize}

O formato deve seguir o modelo:
\begin{verbatim}
Case X: Y nodes not reachable from node Z with TTL = T.
\end{verbatim}

\subsubsection*{Exemplo de Entrada e Saída}

\textbf{Entrada:}
\begin{verbatim}
16
10 15 15 20 20 25 10 30 30 47 47 50 25 45 45 65
15 35 35 55 20 40 50 55 35 40 55 60 40 60 60 65
35 2
35 3
0 0
\end{verbatim}

\textbf{Saída:}
\begin{verbatim}
Case 1: 5 nodes not reachable from node 35 with TTL = 2.
Case 2: 1 nodes not reachable from node 35 with TTL = 3.
\end{verbatim}

\subsubsection*{Restrições}

\begin{itemize}
    \item \( 1 \leq NC \leq 30 \).
    \item Cada rede tem no máximo 30 nós.
    \item Cada consulta consiste em dois inteiros: nó inicial e TTL.
    \item As consultas terminam com o par \( (0,0) \).
\end{itemize}

\subsubsection*{Solução}

A solução emprega um algoritmo de \textbf{Busca em Largura (BFS)} para contar o número de nós alcançáveis a partir de um nó inicial dentro do limite imposto pelo TTL.

\begin{lstlisting}[language=C++]
int bfs(int start, map<int, vector<int>> &adj, int ttl) {
    queue<int> q;
    set<int> vis;
    map<int, int> dist;
    q.push(start);
    vis.insert(start);
    dist[start] = 0;

    while (!q.empty()) {
        int vertex = q.front();
        q.pop();
        if (dist[vertex] < ttl) {
            for (auto neighbor : adj[vertex]) {
                if (vis.find(neighbor) == vis.end()) {
                    q.push(neighbor);
                    vis.insert(neighbor);
                    dist[neighbor] = dist[vertex] + 1;
                }
            }
        }
    }
    return vis.size();
}
\end{lstlisting}

\subsection{Bmail Computer Network}


\subsubsection*{Enunciado}
Era uma vez apenas um roteador na conhecida empresa Bmail. Os anos se passaram e, com o tempo, novos roteadores foram comprados. Cada vez que um novo roteador era adquirido, ele era conectado a um dos roteadores já comprados. Você recebe os valores $p_i$ --- o índice do roteador ao qual o $i$-ésimo roteador foi conectado após ser comprado (ou seja, $p_i < i$).

Agora há $n$ roteadores na Boogle no total. Imprima a sequência de roteadores no caminho do primeiro até o $n$-ésimo roteador.

\subsubsection*{Entrada}
A primeira linha contém um número inteiro $n$ ($2 \le n \le 200000$) --- o número dos roteadores. A linha seguinte contém $n-1$ inteiros 
\[
p_2, p_3, \dots, p_n
\]
($1 \le p_i < i$), onde $p_i$ é igual ao índice do roteador ao qual o $i$-ésimo foi conectado após a compra.

\subsubsection*{Saída}
Imprima o caminho do $1$-ésimo ao $n$-ésimo roteador. O caminho deve começar com o roteador $1$ e terminar com o roteador $n$, com todos os elementos distintos.

\subsubsection*{Exemplos}
\textbf{Entrada:}
\begin{verbatim}
8
1 1 2 2 3 2 5
\end{verbatim}

\textbf{Saída:}
\begin{verbatim}
1 2 5 8 
\end{verbatim}

\textbf{Entrada:}
\begin{verbatim}
6
1 2 3 4 5
\end{verbatim}

\textbf{Saída:}
\begin{verbatim}
1 2 3 4 5 6 
\end{verbatim}

\textbf{Entrada:}
\begin{verbatim}
7
1 1 2 3 4 3
\end{verbatim}

\textbf{Saída:}
\begin{verbatim}
1 3 7 
\end{verbatim}

\subsubsection*{Código em C++}
\begin{lstlisting}[language=C++]

vector<int> bfs(vector<vector<int>> &graph, int start, int end)
{
    queue<int> q;
    unordered_map<int, bool> visited;
    unordered_map<int, int> parent;

    q.push(start);
    visited[start] = true;

    while (!q.empty())
    {
        int node = q.front();
        q.pop();

        if (node == end)
        {
            vector<int> path;
            while (node != start)
            {
                path.push_back(node);
                node = parent[node];
            }
            path.push_back(start);
            reverse(path.begin(), path.end());
            return path;
        }

        for (int neighbor : graph[node])
        {
            if (!visited[neighbor])
            {
                q.push(neighbor);
                visited[neighbor] = true;
                parent[neighbor] = node;
            }
        }
    }
    return {};
}

int main()
{
    ios_base::sync_with_stdio(false);
    cin.tie(0);
    cout.tie(0);
    int n, x;
    cin >> n;
    vector<vector<int>> graph(n + 1);

    for (int i = 2; i <= n; i++)
    {
        cin >> x;
        graph[i].push_back(x);
        graph[x].push_back(i);
    }

    for (auto i : bfs(graph, 1, n))
        cout << i << " ";

    cout << endl;
    return 0;
}
\end{lstlisting}


\subsection{Unlock the Lock}

\subsubsection*{Enunciado}
Mr. Ferdaus criou um tipo especial de cadeado de 4 dígitos, denominado \textit{FeruLock}, mostrado na figura à esquerda. O cadeado sempre exibe um valor de 4 dígitos e possui um código de desbloqueio específico (um valor inteiro). O cadeado é destravado somente quando o código de desbloqueio é exibido. Esse código pode ser rapidamente apresentado com a ajuda de alguns botões especiais disponíveis no cadeado. Cada botão tem um número associado a ele. Quando qualquer um desses botões é pressionado, o número associado é somado ao valor exibido, e assim um novo número é mostrado. O cadeado utiliza sempre os 4 dígitos menos significativos após a adição.

Após criar esse cadeado, Mr. Ferdaus descobriu que também era muito difícil desbloquear o \textit{FeruLock}. Como um bom amigo de Ferdaus, sua tarefa é criar um programa que o ajude a desbloquear o \textit{FeruLock} pressionando esses botões o mínimo número de vezes possível.

\subsubsection*{Entrada}
Haverá no máximo 100 casos de teste. Para cada caso de teste, serão fornecidos 3 números: \(L\), \(U\) e \(R\), onde:
\begin{itemize}
    \item \(L\) (\(0000 \le L \le 9999\)) representa o código atual do cadeado.
    \item \(U\) (\(0000 \le U \le 9999\)) representa o código de desbloqueio.
    \item \(R\) (\(1 \le R \le 10\)) representa o número de botões disponíveis.
\end{itemize}
Em seguida, há \(R\) números (\(0 \le RVi \le 9999\)) em uma única linha, representando o valor dos botões. Os valores de \(L\), \(U\) e \(RVi\) serão sempre apresentados com 4 dígitos (completados com zeros à esquerda, se necessário). A entrada é finalizada quando \(L = U = R = 0\).

\subsubsection*{Saída}
Para cada caso de teste, imprima uma linha de saída que contenha o número do caso seguido pelo mínimo número de pressionamentos de botão necessários para desbloquear o cadeado. Se não for possível desbloquear o cadeado, imprima a mensagem \texttt{Permanently Locked}.

\subsubsection*{Exemplos}

\textbf{Entrada:}
\begin{verbatim}
0000 9999 1
1000
\end{verbatim}

\textbf{Saída:}
\begin{verbatim}
Case 1: Permanently Locked
\end{verbatim}

\textbf{Entrada:}
\begin{verbatim}
0000 9999 1
0001
\end{verbatim}

\textbf{Saída:}
\begin{verbatim}
Case 2: 9999
\end{verbatim}

\textbf{Entrada:}
\begin{verbatim}
5234 1212 3
1023 0101 0001
\end{verbatim}

\textbf{Saída:}
\begin{verbatim}
Case 3: 48
\end{verbatim}

\subsubsection*{Código em C++}
\begin{lstlisting}
void solve()
{
    int l, u, r, cs = 1;
    
    while(cin >> l >> u >> r && (l || u || r)){
        vi rv(r), dist(1e5+10, -1);
        rep(i, 0, r) cin >> rv[i];
        dist[l] = 0;
        queue<int> q;
        q.push(l);
        while(!q.empty()){
            int x = q.front(); q.pop();
            rep(i, 0, r){
                int v = (x + rv[i]) % 10000;
                if(dist[v] == -1){
                    dist[v] = dist[x] + 1;
                    q.push(v);
                }
            }
        }
        cout << "Case " << cs++ << ": ";
        if(dist[u] != -1) cout << dist[u] << endl;
        else cout << "Permanently Locked\n";
    }
}
\end{lstlisting}

\subsection{Word Transformation}

\subsubsection*{Enunciado}
Um quebra-cabeça de palavras comum encontrado em muitos jornais e revistas é a transformação de palavras. Dado uma palavra inicial, pode-se, alterando sucessivamente uma única letra para formar uma nova palavra, construir uma sequência que transforma a palavra original em uma palavra final determinada. Por exemplo, a palavra \texttt{spice} pode ser transformada em \texttt{stock} em quatro etapas segundo a seguinte sequência:
\begin{center}
\texttt{spice, slice, slick, stick, stock}
\end{center}
Cada palavra sucessiva difere da anterior em apenas uma posição de caractere, mantendo o mesmo comprimento.

Dado um dicionário de palavras a partir do qual as transformações podem ser feitas, além de uma lista de palavras iniciais e finais, sua tarefa é determinar o número mínimo de passos na transformação mais curta possível.

\subsubsection*{Entrada}
A primeira linha da entrada contém um inteiro \(N\), indicando o número de conjuntos de teste. Cada conjunto de teste possui duas seções:
\begin{itemize}
    \item \textbf{Dicionário:} Cada palavra é dada em uma linha, terminado por uma linha contendo um asterisco (\texttt{*}). O dicionário pode conter até 200 palavras; todas serão formadas apenas por letras minúsculas e terão no máximo 10 caracteres.
    \item \textbf{Transformações:} Após o dicionário, vêm pares de palavras, uma par por linha, separadas por um espaço. Estes pares representam a palavra inicial e a palavra final de uma transformação. Todas as transformações são possíveis usando o dicionário fornecido, e as palavras inicial e final aparecem no dicionário.
\end{itemize}
Conjuntos de teste consecutivos são separados por uma linha em branco.

\subsubsection*{Saída}
Para cada par de palavras de cada conjunto de teste, imprima uma linha com a palavra inicial, a palavra final e o número mínimo de passos na transformação, separados por um único espaço. Conjuntos de saída consecutivos devem ser separados por uma linha em branco.

\subsubsection*{Exemplo}

\textbf{Entrada:}
\begin{verbatim}
1
dip
lip
mad
map
maple
may
pad
pip
pod
pop
sap
sip
slice
slick
spice
stick
stock
*
spice stock
may pod
\end{verbatim}

\textbf{Saída:}
\begin{verbatim}
spice stock 4
may pod 3
\end{verbatim}

\subsubsection*{Código em C++}
\begin{lstlisting}
void solve()
{
    string s;
    // Cria um conjunto com todas as palavras do dicionário
    set<string> st;
    while(cin >> s && s != "*"){
        st.insert(s);
    }
    string a, b;
    cin.ignore();
    
    // Processa cada par de palavras (transformação)
    while(getline(cin, s) && s != ""){
        stringstream ss(s);
        ss >> a >> b;
        if(a == b){
            cout << a << " " << b << " 0\n";
            continue;
        }
        map<string, int> dist;
        // Fila para a busca em largura (BFS)
        queue<pair<string, int>> q;
        q.push({a, 0});
        dist[a] = 0;
        while(!q.empty()){
            auto [u, d] = q.front();
            q.pop();
            if(u == b){
                cout << a << " " << b << " " << d << endl;
                break;
            }
            // Tenta transformar a palavra u em outra palavra v do dicionário
            for(auto &v : st){
                if(v.size() != u.size()) continue;
                int cnt = 0;
                // Conta quantas posições diferem entre u e v
                for (int i = 0; i < v.size(); i++){
                    if(v[i] != u[i]) cnt++;
                }
                // Se diferem em apenas uma posição, essa é uma transformação válida
                if(cnt == 1 && !dist.count(v)){
                    dist[v] = d + 1;
                    q.push({v, d + 1});
                }
            }
        }
    }
}
\end{lstlisting}


\subsection{Min of Restricted Sum}

\subsubsection*{Enunciado}
Você recebe inteiros \(N, M\) e três sequências inteiras de comprimento \(M\): 
\[
X = \bigl(X_1, X_2, \ldots, X_M\bigr),\quad
Y = \bigl(Y_1, Y_2, \ldots, Y_M\bigr) \quad \text{e} \quad
Z = \bigl(Z_1, Z_2, \ldots, Z_M\bigr).
\]
É garantido que todos os elementos de \(X\) e \(Y\) estão entre \(1\) e \(N\) (inclusive).

Chamamos uma sequência de inteiros não negativos de comprimento \(N\),
\[
A = \bigl(A_1, A_2, \ldots, A_N\bigr),
\]
de \emph{boa sequência} se e somente se, para todo inteiro \(i\) com \(1 \le i \le M\), 
\[
A_{X_i} \oplus A_{Y_i} = Z_i.
\]
Determine se existe uma boa sequência e, se existir, encontre uma que minimize a soma de seus elementos, isto é, que minimize
\[
\sum_{i=1}^{N} A_i.
\]

\paragraph{Notas sobre XOR:} Para inteiros não negativos \(A\) e \(B\), o XOR, denotado por \(A \oplus B\), é definido da seguinte forma: na representação binária, o dígito correspondente à posição \(2^k\) (com \(k \ge 0\)) do resultado é \(1\) se e somente se exatamente um dos dígitos de \(A\) e \(B\) na mesma posição for \(1\); caso contrário, o dígito é \(0\). Por exemplo, \(3 \oplus 5 = 6\) (em binário: \(011 \oplus 101 = 110\)).

\subsubsection*{Restrições}
\begin{itemize}
    \item \(1 \le N \le 2\times 10^5\).
    \item \(0 \le M \le 10^5\).
    \item \(1 \le X_i, Y_i \le N\) para todo \(i\).
    \item \(0 \le Z_i \le 10^9\) para todo \(i\).
    \item Todos os valores de entrada são inteiros.
\end{itemize}

\subsubsection*{Entrada}
A entrada é fornecida a partir da entrada padrão no seguinte formato:
\begin{verbatim}
N M
X_1 Y_1 Z_1
X_2 Y_2 Z_2
.
.
.
X_M Y_M Z_M
\end{verbatim}

\subsubsection*{Saída}
Se nenhuma boa sequência existir, imprima \(-1\). Caso contrário, imprima uma boa sequência que minimize a soma de seus elementos, separando os valores por espaços. Se houver mais de uma resposta com soma mínima, qualquer uma delas será aceita.

\subsubsection*{Exemplo 1}
\textbf{Entrada:}
\begin{verbatim}
3 2
1 3 4
1 2 3
\end{verbatim}
\textbf{Saída:}
\begin{verbatim}
0 3 4
\end{verbatim}

A sequência \(A = (0, 3, 4)\) é uma boa sequência, pois:
\[
A_1 \oplus A_2 = 0 \oplus 3 = 3 \quad \text{e} \quad A_1 \oplus A_3 = 0 \oplus 4 = 4.
\]
Embora outras boas sequências, como \((1,2,5)\) ou \((7,4,3)\), existam, a sequência \((0,3,4)\) possui a menor soma.

\subsubsection*{Exemplo 2}
\textbf{Entrada:}
\begin{verbatim}
3 3
1 3 4
1 2 3
2 3 5
\end{verbatim}
\textbf{Saída:}
\begin{verbatim}
-1
\end{verbatim}

Nenhuma boa sequência existe neste caso.

\subsubsection*{Exemplo 3}
\textbf{Entrada:}
\begin{verbatim}
5 8
4 2 4
2 3 11
3 4 15
4 5 6
3 2 11
3 3 0
3 1 9
3 4 15
\end{verbatim}
\textbf{Saída:}
\begin{verbatim}
0 2 9 6 0
\end{verbatim}

\subsubsection*{Solução}
\begin{lstlisting}[language=C++]
void solve()
{
    int n, m; 
    cin >> n >> m;
    vector<vector<pair<int, int>>> graph(n);
    for (int i = 0; i < m; i++){
        int x, y, z; 
        cin >> x >> y >> z;
        x--, y--;
        graph[x].push_back({y, z});
        graph[y].push_back({x, z});
    }
    vector<int> a(n, -1);
    bool ok = true;
    for (int i = 0; i < n; i++){
        if(a[i] == -1){
            queue<int> q;
            q.push(i);
            a[i] = 0;
            vector<int> cmp;
            cmp.push_back(i);
            while(!q.empty()){
                int u = q.front(); 
                q.pop();
                for(auto [v, w] : graph[u]){
                    if(a[v] == -1){
                        a[v] = a[u] ^ w;
                        q.push(v);
                        cmp.push_back(v);
                    }
                    else if(a[v] != (a[u] ^ w)){
                        ok = false;
                        break;
                    }
                }
                if(!ok) break;
            }
            if(!ok) break;
            int r = 0;
            for (int bit = 0; bit < 32; bit++){
                int mask = 1 << bit;
                int cnt = 0;
                for (auto v : cmp){
                    if(a[v] & mask) cnt++;
                }
                if(cnt > (int)cmp.size() - cnt)
                    r |= mask;
            }
            for (auto v : cmp){
                a[v] ^= r;
            }
        }
        if(!ok) break;
    }
    if(!ok)
        cout << -1 << endl;
    else {
        for(auto x : a)
            cout << x << " ";
        cout << endl;
    }
}
\end{lstlisting}

\subsection{Palindromic Shortest Path}

\subsubsection*{Enunciado}
Temos um grafo direcionado com \(N\) vértices, numerados de \(1, 2, \dots, N\). As informações sobre as arestas são dadas por \(N^2\) caracteres, ou seja, uma matriz onde o elemento \(C_{i,j}\) representa o rótulo da aresta do vértice \(i\) para o vértice \(j\). Cada \(C_{i,j}\) é uma letra minúscula do alfabeto inglês ou o caractere \texttt{-}, indicando que não há aresta entre os vértices \(i\) e \(j\).

Se \(C_{i,j}\) for uma letra, existe exatamente uma aresta direcionada do vértice \(i\) para o vértice \(j\) rotulada como \(C_{i,j}\). Caso \(C_{i,j}\) seja \texttt{-}, não há aresta de \(i\) para \(j\).

Para cada par \((i, j)\) com \(1 \le i, j \le N\), determine o comprimento do caminho mais curto (não necessariamente simples) do vértice \(i\) para o vértice \(j\) cuja concatenação dos rótulos das arestas forma um palíndromo. Se não existir tal caminho, a resposta é \(-1\).

\subsubsection*{Entrada}
A entrada é composta por:
\begin{verbatim}
N
C_{1,1} C_{1,2} ... C_{1,N}
C_{2,1} C_{2,2} ... C_{2,N}
.
.
.
C_{N,1} C_{N,2} ... C_{N,N}
\end{verbatim}
onde cada \(C_{i,j}\) é um caractere que representa o rótulo da aresta do vértice \(i\) para o vértice \(j\), ou \texttt{-} se não existir aresta.

\subsubsection*{Saída}
Imprima uma matriz \(N \times N\) onde o elemento na \(i\)-ésima linha e \(j\)-ésima coluna é o comprimento do caminho mais curto do vértice \(i\) para o vértice \(j\) cuja concatenação de rótulos forma um palíndromo. Caso não exista tal caminho, imprima \(-1\).

\subsubsection*{Restrições}
\[
1 \le N \le 100
\]
Cada \(C_{i,j}\) é uma letra minúscula do alfabeto inglês ou \texttt{-}.

\subsubsection*{Exemplo 1}
\textbf{Entrada:}
\begin{verbatim}
4
ab--
--b-
---a
c---
\end{verbatim}

\textbf{Saída:}
\begin{verbatim}
0 1 2 4
-1 0 1 -1
3 -1 0 1
1 -1 -1 0
\end{verbatim}

\subsubsection*{Exemplo 2}
\textbf{Entrada:}
\begin{verbatim}
5
us---
-st--
--s--
u--s-
---ts
\end{verbatim}

\textbf{Saída:}
\begin{verbatim}
0 1 3 -1 -1
-1 0 1 -1 -1
-1 -1 0 -1 -1
1 3 -1 0 -1
-1 -1 5 1 0
\end{verbatim}

\subsubsection*{Solução}
\begin{lstlisting}[language=C++]
void solve(){
    int n;
    cin >> n;
    vector<string> s(n);
    for(auto &str : s)
        cin >> str;
    
    // Inicializa a matriz de distâncias com INF.
    vvi dist(n, vi(n, INF));
    queue<pii> q;
    
    // Para cada vértice, a distância dele para ele mesmo é 0.
    for (int i = 0; i < n; i++){
        q.push({i, i});
        dist[i][i] = 0;
    }
    
    // Se existir aresta de i para j, a distância é 1.
    for (int i = 0; i < n; i++){
        for (int j = 0; j < n; j++){
            if(i == j || s[i][j] == '-')
                continue;
            q.push({i, j});
            dist[i][j] = 1;
        }
    }
    
    // Propaga as distâncias usando uma busca em largura.
    while(!q.empty()){
        auto [i, j] = q.front();
        q.pop();
        for (int k = 0; k < n; k++){
            for (int l = 0; l < n; l++){
                // Verifica se há aresta de k para i e de j para l,
                // e se os rótulos são iguais.
                if(s[k][i] != '-' && s[j][l] != '-' &&
                   s[j][l] == s[k][i] && dist[k][l] == INF){
                    dist[k][l] = dist[i][j] + 2;
                    q.push({k, l});
                }
            }
        }
    }
    
    // Imprime a matriz de distâncias.
    for (int i = 0; i < n; i++){
        for (int j = 0; j < n; j++){
            cout << (dist[i][j] == INF ? -1 : dist[i][j])
                 << (j == n-1 ? "\n" : " ");
        }
    }
}
\end{lstlisting}