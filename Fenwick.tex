\section{Fenwick}
\subsection{Rotated Inversions}

\subsubsection*{Enunciado}
Você recebe inteiros \(N, M\) e uma sequência de inteiros não negativos de comprimento \(N\):
\[
A = (A_1, A_2, \ldots, A_N).
\]
Para cada inteiro \(k\) com \(0 \le k \le M-1\), defina a sequência
\[
B = (B_1, B_2, \ldots, B_N),
\]
onde
\[
B_i = (A_i + k) \bmod M.
\]
Sua tarefa é calcular o número de inversões na sequência \(B\). O número de inversões de uma sequência é o número de pares \((i,j)\) tais que
\[
1 \le i < j \le N \quad \text{e} \quad B_i > B_j.
\]

\subsubsection*{Restrições}
\begin{itemize}
    \item \(1 \le N, M \le 2 \times 10^5\).
    \item \(0 \le A_i < M\) para \(1 \le i \le N\).
    \item Todos os valores de entrada são inteiros.
\end{itemize}

\subsubsection*{Entrada}
A entrada é fornecida a partir da entrada padrão no seguinte formato:
\begin{verbatim}
N M
A_1 A_2 \dots A_N
\end{verbatim}

\subsubsection*{Saída}
Imprima \(M\) linhas. A \(i\)-ésima linha (para \(1 \le i \le M\)) deve conter a resposta para o caso \(k = i-1\); isto é, o número de inversões na sequência \(B\) formada com \(k = i-1\).

\subsubsection*{Exemplo 1}
\textbf{Entrada:}
\begin{verbatim}
3 3
2 1 0
\end{verbatim}
\textbf{Saída:}
\begin{verbatim}
3
1
1
\end{verbatim}

\subsubsection*{Exemplo 2}
\textbf{Entrada:}
\begin{verbatim}
5 6
5 3 5 0 1
\end{verbatim}
\textbf{Saída:}
\begin{verbatim}
7
3
3
1
1
5
\end{verbatim}

\subsubsection*{Exemplo 3}
\textbf{Entrada:}
\begin{verbatim}
7 7
0 1 2 3 4 5 6
\end{verbatim}
\textbf{Saída:}
\begin{verbatim}
0
6
10
12
12
10
6
\end{verbatim}

\subsubsection*{Solução}
A seguir, uma possível solução em C++ que utiliza uma Árvore de Fenwick (BIT) para contar inversões e aproveita o fato de que as inversões podem ser atualizadas de forma incremental ao rotacionar a sequência.

\begin{lstlisting}[language=C++]
class FenwickTree {
    private:
        int n;
        vector<int> tree;
    public:
        // Cria uma BIT com n posições (0-indexado publicamente)
        FenwickTree(int n) : n(n), tree(n+1, 0) {}

        // Adiciona 'x' na posição p (0-indexada)
        void add(int p, int x) {
            for(++p; p <= n; p += p & -p)
                tree[p] += x;
        }

        // Soma prefixo [0, p)
        int sum(int p) const {
            int res = 0;
            for(; p; p -= p & -p)
                res += tree[p];
            return res;
        }

        // Soma no intervalo [l, r)
        int range_sum(int l, int r) const {
            return sum(r) - sum(l);
        }
};

void solve()
{
    int n, m;
    cin >> n >> m;
    // Para cada valor de A, armazena os índices onde ele ocorre.
    vector<vector<int>> g(m);
    vector<int> a(n);
    for (int i = 0; i < n; i++){
        cin >> a[i];
        g[a[i]].push_back(i);
    }
    
    int ans = 0;
    FenwickTree fenw(m);
    // Contagem de inversões na sequencia A (caso k = 0)
    for (auto x : a) {
        ans += fenw.range_sum(x + 1, m);
        fenw.add(x, 1);
    }
    cout << ans << endl;
    
    // Para cada rotação k de 1 a m-1, atualize o número de inversões
    for (int c = 1; c < m; c++){
        int c1 = 0, c2 = 0;
        // 'g[m-c]' armazena os índices dos elementos iguais a m-c
        auto &grupo = g[m - c];
        for (int i = 0; i < (int)grupo.size(); i++){
            c1 += grupo[i] - i;
            c2 += (n - 1 - grupo[i]) - ((int)grupo.size() - 1 - i);
        }
        ans += c1 - c2;
        cout << ans << endl;
    }
}
\end{lstlisting}

\subsection{Insert}

\subsubsection*{Enunciado}
Há um array vazio \(A\). Para \(i = 1, 2, \dots, N\), execute a seguinte operação em ordem:
\begin{itemize}
    \item Insira o número \(i\) em \(A\) de modo que se torne o \(P_i\)-ésimo elemento a partir do início.
    \begin{itemize}
        \item Mais precisamente, substitua \(A\) pela concatenação dos primeiros \(P_i - 1\) elementos de \(A\), seguido por \(i\), e depois os elementos restantes de \(A\) a partir do \(P_i\)-ésimo elemento, nesta ordem.
    \end{itemize}
\end{itemize}
Ao final, obtenha o array final \(A\).

\subsubsection*{Restrições}
\[
1 \leq N \leq 5 \times 10^5
\]
\[
1 \leq P_i \leq i \quad \text{para } i = 1, 2, \dots, N.
\]
Todos os valores de entrada são inteiros.

\subsubsection*{Entrada}
A entrada é fornecida a partir da entrada padrão no seguinte formato:
\begin{verbatim}
N
P_1 P_2 ... P_N
\end{verbatim}

\subsubsection*{Saída}
Seja o array final 
\[
A = (A_1, A_2, \dots, A_N).
\]
Imprima \(A_1, A_2, \dots, A_N\) nesta ordem, separados por espaços.

\subsubsection*{Exemplo 1}
\textbf{Input:}
\begin{verbatim}
4
1 1 2 1
\end{verbatim}

\textbf{Output:}
\begin{verbatim}
4 2 3 1
\end{verbatim}

\textbf{Explicação:}
\begin{enumerate}
    \item Insira o número \(1\) de modo que se torne o 1º elemento de \(A\). Agora, \(A = (1)\).
    \item Insira o número \(2\) de modo que se torne o 1º elemento de \(A\). Agora, \(A = (2, 1)\).
    \item Insira o número \(3\) de modo que se torne o 2º elemento de \(A\). Agora, \(A = (2, 3, 1)\).
    \item Insira o número \(4\) de modo que se torne o 1º elemento de \(A\). Agora, \(A = (4, 2, 3, 1)\).
\end{enumerate}

\subsubsection*{Exemplo 2}
\textbf{Input:}
\begin{verbatim}
5
1 2 3 4 5
\end{verbatim}

\textbf{Output:}
\begin{verbatim}
1 2 3 4 5
\end{verbatim}


\begin{lstlisting}
    struct BIT {
    int n;
    vector<int> tree;
    BIT(int n): n(n), tree(n+1, 0) {}

    void update(int idx, int delta) {
        for(; idx <= n; idx += idx & -idx)
            tree[idx] += delta;
    }

    int query(int idx) {
        int sum = 0;
        for(; idx; idx -= idx & -idx)
            sum += tree[idx];
        return sum;
    }

    int find(int k) {
        int idx = 0;
        for (int bit = 1 << 20; bit; bit >>= 1) {
            int next = idx + bit;
            if(next <= n && tree[next] < k) {
                k -= tree[next];
                idx = next;
            }
        }
        return idx+1;
    }
};

void solve() {
    int n;
    cin >> n;
    vi p(n), a(n+1, 0);
    for (auto &x : p) cin >> x;

    BIT bit(n);
    rep (i, 1, n+1)
        bit.update(i, 1);


    for(int i = n; i >= 1; i--) {
        int pos = bit.find(p[i-1]);
        a[pos] = i;
        bit.update(pos, -1);
    }

    rep(i,1,n+1){
        cout << a[i] << " ";
    }
    cout << endl;
}
\end{lstlisting}