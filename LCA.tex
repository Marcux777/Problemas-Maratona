\section{Lowest Common Ancestor}
\subsection{Ants Colony}

\subsubsection*{Enunciado}
Um grupo de formigas está realmente orgulhoso porque construiu uma colônia magnífica e grande. Contudo, o tamanho enorme se tornou um problema, pois muitas formigas não conhecem o caminho entre algumas partes da colônia e precisam desesperadamente da sua ajuda!

A colônia foi construída como uma série de \(n\) formigueiros, conectados por túneis. As formigas numeraram os formigueiros sequencialmente à medida que os construíam. O primeiro formigueiro, numerado como \(0\), não exigiu nenhum túnel; mas, para cada formigueiro subsequente, de \(1\) a \(n-1\), foi construído exatamente um túnel que conecta o novo formigueiro a um dos formigueiros já existentes. Esse túnel é suficiente para permitir que qualquer formiga vá de um formigueiro a outro, possivelmente passando por outros formigueiros ao longo do caminho.

Seu trabalho é, dada a estrutura da colônia e um conjunto de consultas, calcular para cada consulta o caminho mais curto entre dois formigueiros. O comprimento de um caminho é definido como a soma dos comprimentos de todos os túneis percorridos.

\subsubsection*{Entrada}
Cada caso de teste é fornecido em várias linhas:
\begin{itemize}
    \item A primeira linha contém um inteiro \(n\) (\(2 \le n \le 10^5\)), o número de formigueiros da colônia.
    \item Cada uma das próximas \(n-1\) linhas contém dois inteiros: \(A[i]\) e \(L[i]\) (para \(i = 1,2,\dots,n-1\)), indicando que o formigueiro \(i\) foi conectado diretamente ao formigueiro \(A[i]\) com um túnel de comprimento \(L[i]\) (\(0 \le A[i] \le i-1\) e \(1 \le L[i] \le 10^9\)).
    \item A próxima linha contém um inteiro \(q\) (\(1 \le q \le 10^5\)), o número de consultas.
    \item Cada uma das próximas \(q\) linhas contém dois inteiros distintos \(s\) e \(t\) (\(0 \le s, t \le n-1\)), representando os formigueiros de origem e destino, respectivamente.
\end{itemize}
O último caso de teste é seguido por uma linha contendo um zero.

\subsubsection*{Saída}
Para cada caso de teste, imprima uma única linha contendo \(q\) inteiros, cada um representando o comprimento do caminho mais curto entre o par de formigueiros da consulta, na mesma ordem em que as consultas foram fornecidas.

\subsubsection*{Exemplo de Entrada}
\begin{verbatim}
6
0 8
1 7
1 9
0 3
4 2
4
2 3
5 2
1 4
0 3
2
0 1
2
1 0
0 1
6
0 1000000000
1 1000000000
2 1000000000
3 1000000000
4 1000000000
1
5 0
0
\end{verbatim}

\subsubsection*{Exemplo de Saída}
\begin{verbatim}
16 20 11 17
1 1
5000000000
\end{verbatim}


\begin{lstlisting}
int depth[MAXN], parent[MAXN][LOG], dist[MAXN];
vector<pair<int, int>> g[MAXN];

void dfs(int u, int par) {
    parent[u][0] = par;
    for(int i = 1; i < LOG; i++) {
        if(parent[u][i-1] != -1) {
            parent[u][i] = parent[parent[u][i-1]][i-1];
        }
    }
    for(auto &it : g[u]) {
        if(it.first == par) continue;
        depth[it.first] = depth[u] + 1;
        dist[it.first] = dist[u] + it.second;
        dfs(it.first, u);
    }
}

int lca(int u, int v) {
    if(depth[u] < depth[v]) swap(u, v);
    for(int i = LOG - 1; i >= 0; i--) {
        if(parent[u][i] != -1 && depth[parent[u][i]] >= depth[v]) {
            u = parent[u][i];
        }
    }
    if(u == v) return u;
    for(int i = LOG - 1; i >= 0; i--) {
        if(parent[u][i] != parent[v][i]) {
            u = parent[u][i];
            v = parent[v][i];
        }
    }
    return parent[u][0];
}

signed main() {
    ios_base::sync_with_stdio(false);
    cin.tie(NULL);
    cout.tie(NULL);
    //freopen("saida.txt", "w", stdout);
    int n;
    while(cin >> n && n) {
        for(int i = 0; i < n; i++) g[i].clear();
        memset(parent, -1, sizeof(parent));
        
        for(int i = 1; i < n; i++) {
            int a, w;
            cin >> a >> w;
            g[i].push_back({a, w});
            g[a].push_back({i, w});
        }
        
        depth[0] = dist[0] = 0;
        dfs(0, -1);
        
        int q;
        cin >> q;
        vector<int>resp;
        while(q--) {
            int s, t;
            cin >> s >> t;
            int ancestor = lca(s, t);
            resp.push_back(dist[s] + dist[t] - 2 * dist[ancestor]);
        }
        for(int i = 0; i <resp.size(); i++){
            cout << resp[i] << (i == resp.size() - 1 ? '\n' : ' ');
        }
    }
\end{lstlisting}