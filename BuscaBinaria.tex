\section{Busca Binaria}

\subsection{Smooth Occlusion}

\subsubsection*{Enunciado}
Takahashi tem \(2N\) dentes: \(N\) dentes superiores e \(N\) dentes inferiores.
O comprimento do \(i\)-ésimo dente superior da esquerda é \(U_i\) e o do \(i\)-ésimo dente inferior da esquerda é \(D_i\) (para \(1 \le i \le N\)).
Os dentes estão considerados "bem ajustados" se ambas as seguintes condições forem satisfeitas:
\begin{enumerate}
    \item Existe um inteiro \(H\) tal que, para todo \(i\) com \(1 \le i \le N\),
    \[
    U_i + D_i = H.
    \]
    \item Para todo \(i\) com \(1 \le i < N\),
    \[
    \lvert U_i - U_{i+1} \rvert \le X.
    \]
\end{enumerate}

Takahashi não pode aumentar os comprimentos dos dentes, apenas reduzi-los.
Para isso, ele dispõe de uma máquina de moagem de dentes: a cada operação, pagando 1 iene, ele pode escolher exatamente um dente cujo comprimento seja positivo e reduzir seu comprimento em 1.
Encontre o valor total mínimo que ele precisa pagar para que seus dentes se ajustem bem.

\subsubsection*{Restrições}
\begin{itemize}
    \item \(2 \le N \le 2 \times 10^5\).
    \item \(1 \le X \le 10^9\).
    \item \(1 \le U_i, D_i \le 10^9\) para \(1 \le i \le N\).
\end{itemize}

\subsubsection*{Entrada}
A entrada é fornecida a partir da entrada padrão no seguinte formato:
\begin{verbatim}
N X
U_1 D_1
U_2 D_2
.
.
.
U_N D_N
\end{verbatim}

\subsubsection*{Saída}
Imprima uma única linha contendo o custo total mínimo que Takahashi precisa pagar.

\subsubsection*{Exemplo}
\textbf{Entrada:}
\begin{verbatim}
4 3
3 1
4 1
5 9
2 6
\end{verbatim}
\textbf{Saída:}
\begin{verbatim}
15
\end{verbatim}

Na entrada do exemplo, a soma total dos comprimentos dos dentes é
\[
(3+1)+(4+1)+(5+9)+(2+6)=4+5+14+8=31.
\]
Se for possível ajustar os dentes para que cada par some, por exemplo, \(H=4\), o custo será
\[
31 - 4\times4 = 31-16 = 15.
\]

\subsubsection*{Solução}
\begin{lstlisting}[language=C++]
void solve()
{
    int n, x; 
    cin >> n >> x;
    vector<int> up(n), dw(n);
    int total = 0;
    int hmax = LINF;
    for (int i = 0; i < n; i++){
        cin >> up[i] >> dw[i];
        total += up[i] + dw[i];
        hmax = min(hmax, up[i] + dw[i]);
    }

    auto f = [&](int h) -> bool {
        int l = max(0LL, h - (long long)dw[0]);
        int r = min(h, up[0]);
        for (int i = 1; i < n; i++){
            int nl = max(0LL, h - (long long)dw[i]);
            int nr = min(h, up[i]);
            if (max(nl, l - x) > min(nr, r + x))
                return false;
            l = max(nl, l - x);
            r = min(nr, r + x);
        }
        return true;
    };

    int lo = 0, hi = hmax;
    while(lo <= hi){
        int mid = (lo + hi) / 2;
        if(f(mid))
            lo = mid + 1;
        else
            hi = mid - 1;
    }
    cout << total - n * (lo - 1) << endl;
}
\end{lstlisting}
