\section{DFS}
\subsection{Protect Sheep}
Bob é um fazendeiro. Ele tem um grande pasto com muitas ovelhas. Recentemente, ele perdeu algumas delas devido a ataques de lobos. Ele decidiu então colocar alguns cães pastores de forma que todas as suas ovelhas estejam protegidas.

O pasto é um retângulo composto por \( R \times C \) células. Cada célula está vazia, contém uma ovelha, um lobo ou um cão. Ovelhas e cães sempre permanecem no lugar, mas os lobos podem vagar livremente pelo pasto, movendo-se repetidamente para a esquerda, direita, para cima ou para baixo para uma célula vizinha. Quando um lobo entra em uma célula com uma ovelha, ele a consome. No entanto, nenhum lobo pode entrar em uma célula com um cão.

Inicialmente não há cães. Coloque cães no pasto de forma que nenhum lobo possa alcançar qualquer ovelha, ou determine que isso é impossível. Note que, como você tem muitos cães, não é necessário minimizar o número deles.

\subsubsection*{Entrada}
A primeira linha contém dois inteiros \( R \) (\(1 \le R \le 500\)) e \( C \) (\(1 \le C \le 500\)), indicando o número de linhas e o número de colunas, respectivamente.

Cada uma das \( R \) linhas seguintes é uma string composta por exatamente \( C \) caracteres, representando uma linha do pasto. Aqui:
\begin{itemize}
    \item \texttt{S} significa uma ovelha,
    \item \texttt{W} significa um lobo,
    \item \texttt{.} significa uma célula vazia.
\end{itemize}

\subsubsection*{Saída}
Se for impossível proteger todas as ovelhas, imprima uma única linha com a palavra \texttt{Não}.

Caso contrário, imprima uma linha com a palavra \texttt{Sim}. Em seguida, imprima \( R \) linhas, representando o pasto após a colocação dos cães. Novamente:
\begin{itemize}
    \item \texttt{S} significa uma ovelha,
    \item \texttt{W} significa um lobo,
    \item \texttt{D} significa um cão,
    \item \texttt{.} significa uma célula vazia.
\end{itemize}

Não é permitido mover, remover ou adicionar uma ovelha ou um lobo. Se houver várias soluções, você pode imprimir qualquer uma delas. Você não precisa minimizar o número de cães.

\subsubsection*{Exemplo 1}

\textbf{Input:}
\begin{verbatim}
6 6
..S...
..S.W.
.S....
..W...
...W..
......
\end{verbatim}

\textbf{Output:}
\begin{verbatim}
Yes
..SD..
..SDW.
.SD...
.DW...
DD.W..
......
\end{verbatim}

\subsubsection*{Exemplo 2}

\textbf{Input:}
\begin{verbatim}
1 2
SW
\end{verbatim}

\textbf{Output:}
\begin{verbatim}
No
\end{verbatim}

\subsubsection*{Exemplo 3}

\textbf{Input:}
\begin{verbatim}
5 5
.S...
...S.
S....
...S.
.S...
\end{verbatim}

\textbf{Output:}
\begin{verbatim}
Yes
.S...
...S.
S.D..
...S.
.S...
\end{verbatim}

\subsubsection*{Observação}
No primeiro exemplo, podemos dividir o pasto em duas metades, uma contendo lobos e outra contendo ovelhas. Observe que a ovelha em (2,1) está segura, pois os lobos não podem se mover diagonalmente.

No segundo exemplo, não há espaços vazios para colocar cães que protegeriam a única ovelha.

No terceiro exemplo, não há lobos, então a tarefa é muito fácil. Colocamos um cão no centro para observar a tranquilidade do prado, mas a solução estaria correta mesmo sem ele.

\subsubsection*{Solução}

\begin{lstlisting}[language=C++]

// Declaração global para a matriz de visitados
vector<vector<bool>> vis;
bool aux;

void floodfillS(vector<string> &g, int x, int y)
{
    if (x < 0 || x >= g.size() || y < 0 || y >= g[0].size())
        return;
    if (vis[x][y])
        return;
    vis[x][y] = true;

    if (g[x][y] == 'S')
    {
        if (x > 0 && g[x - 1][y] == '.')
            g[x - 1][y] = 'D';
        if (x < g.size() - 1 && g[x + 1][y] == '.')
            g[x + 1][y] = 'D';
        if (y > 0 && g[x][y - 1] == '.')
            g[x][y - 1] = 'D';
        if (y < g[0].size() - 1 && g[x][y + 1] == '.')
            g[x][y + 1] = 'D';
    }

    floodfillS(g, x - 1, y);
    floodfillS(g, x + 1, y);
    floodfillS(g, x, y - 1);
    floodfillS(g, x, y + 1);
}

void floodfillW(vector<string> &g, int x, int y)
{
    if (x < 0 || x >= g.size() || y < 0 || y >= g[0].size())
        return;
    if (vis[x][y] || g[x][y] == 'D')
        return;
    vis[x][y] = true;

    if (g[x][y] == 'S')
        aux = true;

    floodfillW(g, x - 1, y);
    floodfillW(g, x + 1, y);
    floodfillW(g, x, y - 1);
    floodfillW(g, x, y + 1);
}

int main()
{
    ios_base::sync_with_stdio(false);
    cin.tie(0);
    cout.tie(0);
    
    int r, c;
    cin >> r >> c;
    vector<string> g(r);
    vis.assign(r, vector<bool>(c, false));

    for (auto &i : g)
        cin >> i;

    // Coloca cães ao redor de cada ovelha
    for (int i = 0; i < r; i++)
    {
        for (int j = 0; j < c; j++)
        {
            if (g[i][j] == 'S')
            {
                floodfillS(g, i, j);
            }
        }
    }
    
    vis.assign(r, vector<bool>(c, false));
    aux = false;
    
    // Verifica se algum lobo pode alcançar uma ovelha
    for (int i = 0; i < r; i++)
    {
        for (int j = 0; j < c; j++)
        {
            if (g[i][j] == 'W')
            {
                floodfillW(g, i, j);
            }
        }
    }
    
    cout << (aux ? "No\n" : "Yes\n");
    if (!aux)
    {
        for (auto s : g)
        {
            cout << s << "\n";
        }
    }
    return 0;
}
\end{lstlisting}

\subsection{Badge}

\subsubsection*{Enunciado}
No Verão da Escola de Informática, se um aluno não se comporta bem, os professores fazem um buraco em sua insígnia. E hoje um dos professores pegou um grupo de \( n \) alunos fazendo mais uma travessura. 

Vamos supor que todos esses alunos são numerados de \( 1 \) a \( n \). O professor chegou ao aluno \( a \) e fez um buraco em sua insígnia. No entanto, o aluno afirmou que o verdadeiro culpado é outro aluno, o \( p_a \). Depois disso, o professor chegou ao aluno \( p_a \) e fez um buraco em sua insígnia também. O aluno respondeu que o verdadeiro culpado era o aluno \( p_{p_a} \). 

Esse processo continuou por um tempo, mas, como o número de alunos era finito, eventualmente o professor chegou ao aluno que já tinha um buraco em sua insígnia. Em seguida, o professor fez um segundo buraco na insígnia desse aluno e decidiu que havia terminado com o processo, indo para a sauna.

Você não sabe qual foi o primeiro aluno pego pelo professor. Contudo, você conhece todos os números \( p_1, p_2, \dots, p_n \) (\( 1 \le p_i \le n \)), onde \( p_i \) indica o aluno denunciado ao professor pelo aluno \( i \).

Sua tarefa é descobrir, para cada aluno \( a \), quem seria o aluno com dois buracos na insígnia, se o primeiro aluno pego fosse o \( a \).

\subsubsection*{Input}
A primeira linha da entrada contém o único inteiro \( n \) (\( 1 \le n \le 1000 \)) --- o número de alunos travessos.

A segunda linha contém \( n \) inteiros \( p_1, p_2, \dots, p_n \) (\( 1 \le p_i \le n \)), onde \( p_i \) indica o aluno denunciado pelo aluno \( i \).

\subsubsection*{Output}
Para cada aluno \( a \) de \( 1 \) a \( n \), imprima qual aluno receberia dois buracos na insígnia, se o primeiro aluno pego fosse o \( a \).

\subsubsection*{Sample 1}

\textbf{Input:}
\begin{verbatim}
3
2 3 2
\end{verbatim}

\textbf{Output:}
\begin{verbatim}
2 2 3
\end{verbatim}

\subsubsection*{Sample 2}

\textbf{Input:}
\begin{verbatim}
3
1 2 3
\end{verbatim}

\textbf{Output:}
\begin{verbatim}
1 2 3
\end{verbatim}

\subsubsection*{Note}
A imagem corresponde ao primeiro exemplo do caso de teste.

\medskip
\textbf{Quando \( a = 1 \):} O professor vai aos alunos \( 1 \), \( 1 \) e \( 2 \), nessa ordem, e o aluno \( 2 \) é quem recebe um segundo buraco em sua insígnia.

\medskip
\textbf{Quando \( a = 2 \):} O professor vai aos alunos \( 2 \), \( 3 \) e \( 2 \), e o aluno \( 2 \) recebe um segundo buraco em sua insígnia.

\medskip
\textbf{Quando \( a = 3 \):} O professor visitará os alunos \( 3 \), \( 2 \) e \( 3 \), com o aluno \( 3 \) recebendo um segundo buraco em sua insígnia.

\medskip
No segundo caso de teste, não importa com quem o professor comece; esse aluno seria sempre o que receberia o segundo buraco em sua insígnia.

\subsubsection*{Solução}

\begin{lstlisting}[language=C++]
vector<bool> vis;
queue<int> q;

void dfs(int u, vector<vector<int>> &g)
{
    vis[u] = true;
    for (auto v : g[u])
    {
        if (!vis[v])
            dfs(v, g);
        else
        {
            q.push(v);
            break;
        }
    }
}

int main()
{
    ios_base::sync_with_stdio(false);
    cin.tie(0);
    cout.tie(0);
    
    int n, x;
    cin >> n;
    vector<vector<int>> graph(n + 1);
    for (int i = 1; i <= n; i++)
    {
        cin >> x;
        graph[i].push_back(x);
    }

    for (int i = 1; i <= n; i++)
    {
        vis.assign(n + 1, false);
        dfs(i, graph);
    }
    while (!q.empty())
    {
        cout << q.front() << " ";
        q.pop();
    }
    return 0;
}
\end{lstlisting}

\subsection{Building Roads}

Byteland tem \( n \) cidades e \( m \) estradas entre elas. O objetivo é construir novas estradas para que haja uma rota entre quaisquer duas cidades.

Sua tarefa é descobrir o número mínimo de estradas necessárias e também determinar quais estradas devem ser construídas.

\subsubsection*{Entrada}
A primeira linha da entrada contém dois inteiros \( n \) e \( m \): o número de cidades e o número de estradas, respectivamente. As cidades são numeradas de \( 1, 2, \dots, n \).

Depois, há \( m \) linhas descrevendo as estradas. Cada linha contém dois inteiros \( a \) e \( b \), indicando que há uma estrada entre as cidades \( a \) e \( b \).

Uma estrada sempre conecta duas cidades diferentes, e há no máximo uma estrada entre quaisquer duas cidades.

\subsubsection*{Saída}
Primeiro, imprima um inteiro \( k \): o número de estradas necessárias para conectar todas as cidades.

Em seguida, imprima \( k \) linhas que descrevem as novas estradas a serem construídas. Você pode imprimir qualquer solução válida.

\subsubsection*{Restrições}
\[
1 \le n \le 10^5
\]
\[
1 \le m \le 2 \cdot 10^5
\]
\[
1 \le a,b \le n
\]

\subsubsection*{Exemplo}
\textbf{Input:}
\begin{verbatim}
4 2
1 2
3 4
\end{verbatim}

\textbf{Output:}
\begin{verbatim}
1
2 3
\end{verbatim}

\subsubsection*{Solução}
\begin{lstlisting}[language=C++]
void dfs(int u, vector<vector<int>> &graph, vector<bool> &vis)
{
    vis[u] = true;
    for (int v : graph[u])
    {
        if (!vis[v])
        {
            dfs(v, graph, vis);
        }
    }
    u;
}
 
int main()
{
    ios::sync_with_stdio(false);
    cin.tie(nullptr);
    cout.tie(nullptr);
    int n, m;
    cin >> n >> m;
    vector<vector<int>> graph(n + 1);
    vector<bool> vis(n + 1, false);
    vector<int> roads;
    for (int i = 0; i < m; i++)
    {
        int x, y;
        cin >> x >> y;
        graph[x].push_back(y);
        graph[y].push_back(x);
    }
    for (int i = 1; i <= n; i++)
    {
        if (!vis[i])
        {
            roads.push_back(i);
            dfs(i, graph, vis);
        }
    }
    cout << roads.size() - 1 << endl;
    for (int i = 0; i < roads.size() - 1; i++)
    {
        cout << roads[i] << " " << roads[i + 1] << endl;
    }
}    
\end{lstlisting}

\subsection{Minimum XOR Path}

\subsubsection*{Enunciado}
Você recebe um grafo não direcionado, simples e conectado com \(N\) vértices numerados de \(1\) a \(N\) e \(M\) arestas numeradas de \(1\) a \(M\). A aresta \(i\) conecta os vértices \(u_i\) e \(v_i\) e possui um rótulo \(w_i\). Entre todos os caminhos simples (caminhos que não passam pelo mesmo vértice mais de uma vez) do vértice \(1\) ao vértice \(N\), encontre o mínimo XOR dos rótulos das arestas presentes no caminho.

\paragraph{Notas sobre XOR:}
Para inteiros não negativos \(A\) e \(B\), o XOR, denotado por \(A \oplus B\), é definido da seguinte forma: na representação binária de \(A\) e \(B\), o dígito na posição correspondente a \(2^k\) (para \(k \ge 0\)) do resultado é \(1\) se, e somente se, exatamente um dos dígitos na mesma posição de \(A\) e \(B\) for \(1\); caso contrário, o dígito é \(0\). Por exemplo, \(3 \oplus 5 = 6\) (em binário: \(011 \oplus 101 = 110\)).

\subsubsection*{Restrições}
\begin{itemize}
    \item \(2 \leq N \leq 10\).
    \item \(N-1 \leq M \leq \frac{N(N-1)}{2}\).
    \item \(1 \leq u_i < v_i \leq N\).
    \item \(0 \leq w_i < 2^{60}\).
    \item O grafo é não direcionado, simples e conectado.
    \item Todos os valores de entrada são inteiros.
\end{itemize}

\subsubsection*{Entrada}
A entrada é fornecida a partir da entrada padrão no seguinte formato:
\begin{verbatim}
N M
u_1 v_1 w_1
u_2 v_2 w_2
.
.
.
u_M v_M w_M
\end{verbatim}

\subsubsection*{Saída}
Imprima a resposta: o mínimo XOR dos rótulos das arestas dentre todos os caminhos simples do vértice \(1\) ao vértice \(N\).

\subsubsection*{Exemplo 1}
\textbf{Entrada:}
\begin{verbatim}
4 4
1 2 3
2 4 5
1 3 4
3 4 7
\end{verbatim}
\textbf{Saída:}
\begin{verbatim}
3
\end{verbatim}

Existem dois caminhos simples do vértice \(1\) ao vértice \(4\):
\begin{itemize}
    \item \(1 \to 2 \to 4\) com XOR dos rótulos: \(3 \oplus 5 = 6\).
    \item \(1 \to 3 \to 4\) com XOR dos rótulos: \(4 \oplus 7 = 3\).
\end{itemize}
Portanto, a resposta é \(3\).

\subsubsection*{Exemplo 2}
\textbf{Entrada:}
\begin{verbatim}
4 3
1 2 1
2 3 2
3 4 4
\end{verbatim}
\textbf{Saída:}
\begin{verbatim}
7
\end{verbatim}

\subsubsection*{Exemplo 3}
\textbf{Entrada:}
\begin{verbatim}
7 10
1 2 726259430069220777
1 4 988687862609183408
1 5 298079271598409137
1 6 920499328385871537
1 7 763940148194103497
2 4 382710956291350101
3 4 770341659133285654
3 5 422036395078103425
3 6 472678770470637382
5 7 938201660808593198
\end{verbatim}
\textbf{Saída:}
\begin{verbatim}
186751192333709144
\end{verbatim}

\subsubsection*{Solução}
\begin{lstlisting}[language=C++]
void solve()
{
    int n, m; 
    cin >> n >> m;
    vector<vector<pair<int, long long>>> graph(n+1);
    for (int i = 0; i < m; i++){
        int x, y; 
        long long w; 
        cin >> x >> y >> w;
        graph[x].push_back({y, w});
        graph[y].push_back({x, w});
    }
    vector<int> vis(n+1, 0);
    long long ans = LLONG_MAX;
    function<void(int, long long)> dfs = [&](int u, long long curr) {
        vis[u] = 1;
        if(u == n)
            ans = min(ans, curr);
        for(auto [v, w] : graph[u]){
            if(!vis[v])
                dfs(v, curr ^ w);
        }
        vis[u] = 0;
    };
    dfs(1, 0);
    cout << ans << endl;
}
\end{lstlisting}

