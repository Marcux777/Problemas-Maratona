\section{Minimum Spanning Tree}

\subsection{Transportation system.}

\subsubsection*{Enunciado}
No país de Graphland, existem muitas cidades, mas não há estradas. O governo federal deseja mudar essa situação e planeja construir estradas e ferrovias de forma que todas as cidades do país fiquem conectadas por meio deste novo sistema de transporte. Para tornar o sistema mais eficiente, Graphland construirá apenas estradas entre cidades do mesmo estado e utilizará ferrovias para conectar cidades que estejam em estados diferentes. Para os fins deste problema, considere que se a distância entre quaisquer duas cidades for, no máximo, \(r\), então elas pertencem ao mesmo estado.

Além disso, para minimizar os custos de construção, o governo pretende erguer apenas a extensão mínima necessária de estradas e ferrovias, de modo que exista um caminho entre qualquer par de cidades em todo o país. Você foi contratado para determinar qual é a rede de transporte ótima que Graphland deve construir.

\subsubsection*{Entrada}
A primeira linha da entrada contém um inteiro \(T\) \((1 \le T \le 20)\), o número de casos de teste.  
Para cada caso de teste:
\begin{itemize}
    \item A primeira linha contém dois inteiros, \(n\) \((1 \le n \le 1000)\) – o número de cidades que compõem Graphland, e \(r\) \((0 \le r \le 40000)\) – o valor limite para determinar se duas cidades estão no mesmo estado.
    \item As \(n\) linhas seguintes contêm duas coordenadas inteiras \(x\) e \(y\) \(( -10000 \le x, y \le 10000 )\), representando o plano cartesiano de cada cidade em Graphland.
\end{itemize}

\subsubsection*{Saída}
Para cada caso de teste, o programa deve imprimir uma única linha contendo:
\begin{itemize}
    \item O identificador do caso, no formato \texttt{Case \#i:} (onde \(i\) começa em 1 e incrementa para cada novo caso),
    \item O número de estados em Graphland,
    \item A extensão mínima (arredondada para o inteiro mais próximo) de estradas que devem ser construídas,
    \item A extensão mínima (arredondada para o inteiro mais próximo) de ferrovias que devem ser construídas.
\end{itemize}

\subsubsection*{Observação}
Note que, pela definição, se \(A\) e \(B\) estão no mesmo estado, e \(B\) e \(C\) estão no mesmo estado, então \(A\) e \(C\) também estão no mesmo estado.

\subsubsection*{Exemplo de Entrada}
\begin{verbatim}
3
3 100
0 0
1 0
2 0
3 1
0 0
100 0
200 0
4 20
0 0
40 30
30 30
10 10
\end{verbatim}

\subsubsection*{Exemplo de Saída}
\begin{verbatim}
Case #1: 1 2 0
Case #2: 3 0 200
Case #3: 2 24 28
\end{verbatim}


\begin{lstlisting}
class DSU
{
    int *parent;
    int *rank;

public:
    DSU(int n)
    {
        parent = new int[n];
        rank = new int[n];

        for (int i = 0; i < n; i++)
        {
            parent[i] = -1;
            rank[i] = 1;
        }
    }

    int find(int i)
    {
        if (parent[i] == -1)
            return i;

        return parent[i] = find(parent[i]);
    }

    void unite(int x, int y)
    {
        int s1 = find(x);
        int s2 = find(y);

        if (s1 != s2)
        {
            if (rank[s1] < rank[s2])
            {
                parent[s1] = s2;
            }
            else if (rank[s1] > rank[s2])
            {
                parent[s2] = s1;
            }
            else
            {
                parent[s2] = s1;
                rank[s1] += 1;
            }
        }
    }
};

class Graph
{
    vector<vector<int>> edgelist;
    int V;
    int r;

public:
    Graph(int V, int r)
    {
        this->V = V;
        this->r = r;
    }
    void addEdge(int x, int y, int w)
    {
        edgelist.push_back({w, x, y});
    }

    vector<double> kruskals_mst()
    {
        sort(edgelist.begin(), edgelist.end());

        DSU s(V);
        double rail = 0;
        double road = 0;
        double estado = 1;
        for (auto edge : edgelist)
        {
            int w = edge[0];
            int x = edge[1];
            int y = edge[2];

            if (s.find(x) != s.find(y))
            {
                s.unite(x, y);
                if (w > r * r)
                {
                    rail += sqrt(w);
                    estado++;
                }
                else
                {
                    road += sqrt(w);
                }
            }
        }
        return {road, rail, estado};
    }
};

int dist(pair<int, int> a, pair<int, int> b)
{
    return ((a.first - b.first) * (a.first - b.first) + (a.second - b.second) * (a.second - b.second));
}

signed main()
{
    ios_base::sync_with_stdio(true);
    cin.tie(0);
    cout.tie(0);
    // freopen("saida.txt", "w", stdout);
    int t, caso = 0;
    cin >> t;
    while (t--)
    {
        int n, r;
        cin >> n >> r;
        vector<pair<int, int>> cidades(n);
        for (int i = 0; i < n; i++)
        {
            int x, y;
            cin >> x >> y;
            cidades[i] = {x, y};
        }
        Graph graph(n, r);
        for (int i = 0; i < n; i++)
        {
            for (int j = 0; j < n; j++)
            {

                int w = dist(cidades[i], cidades[j]);
                graph.addEdge(i, j, w);
            }
        }
        cout << fixed << setprecision(0);
        cout << "Case #" << ++caso << ": ";
        vector<double> c = graph.kruskals_mst();
        cout << c[2] << " ";
        cout << c[0] << " " << c[1] << endl;
    }
}
\end{lstlisting}