\section{DSU}
\subsection{Cables and Servers}

\subsubsection*{Enunciado}
Existem \(N\) servidores numerados de \(1\) a \(N\) e \(M\) cabos numerados de \(1\) a \(M\).  
Cada cabo \(i\) conecta os servidores \(A_i\) e \(B_i\) bidirecionalmente.  
Realizando a seguinte operação (possivelmente zero vezes), faça com que todos os servidores estejam conectados via cabos:
\begin{quote}
Operação: Escolha um cabo e reconecte uma de suas extremidades a um servidor diferente.
\end{quote}
Encontre o número mínimo de operações necessárias e imprima uma sequência de operações que atinja esse mínimo.

\subsubsection*{Restrições}
\[
2 \leq N \leq 2 \times 10^5
\]
\[
N-1 \leq M \leq 2 \times 10^5
\]
\[
1 \leq A_i, B_i \leq N \quad \text{para } i=1,2,\dots,M.
\]
Todos os valores de entrada são inteiros.

\subsubsection*{Entrada}
A entrada é fornecida a partir da entrada padrão no seguinte formato:
\begin{verbatim}
N M
A_1 B_1
A_2 B_2
...
A_M B_M
\end{verbatim}

\subsubsection*{Saída}
Seja \(K\) o número mínimo de operações necessárias. Imprima \(K+1\) linhas:
\begin{itemize}
    \item A primeira linha deve conter o valor \(K\).
    \item A \((i+1)\)-ésima linha (para \(i=0,1,\dots,K-1\)) deve conter três inteiros separados por espaço: o número do cabo escolhido na \(i\)-ésima operação, o número do servidor que estava originalmente conectado naquela extremidade, e o número do servidor ao qual está conectado após a operação.
\end{itemize}
Se houver múltiplas soluções válidas, qualquer uma delas será aceita.

\subsubsection*{Exemplo 1}
\textbf{Input:}
\begin{verbatim}
4 5
1 1
1 2
2 1
3 4
4 4
\end{verbatim}

\textbf{Output:}
\begin{verbatim}
1
1 1 3
\end{verbatim}

*Explicação:* Reconectando a extremidade do cabo \(1\) que estava conectado ao servidor \(1\) para o servidor \(3\), os servidores passam a estar conectados.

\subsubsection*{Exemplo 2}
\textbf{Input:}
\begin{verbatim}
4 3
3 4
4 1
1 2
\end{verbatim}

\textbf{Output:}
\begin{verbatim}
0
\end{verbatim}

*Nenhuma operação é necessária.*

\subsubsection*{Exemplo 3}
\textbf{Input:}
\begin{verbatim}
5 4
3 3
3 3
3 3
3 3
\end{verbatim}

\textbf{Output:}
\begin{verbatim}
4
1 3 5
2 3 4
3 3 2
4 3 1
\end{verbatim}

*Esta sequência de operações reconecta os cabos de forma que, ao final, todos os servidores estejam conectados.*


\begin{lstlisting}
class DSU {
    vector<int> p, sz;
public:
    DSU(int n) {
        p.resize(n);
        sz.resize(n, 1);
        iota(p.begin(), p.end(), 0);
    }

    int find(int x) {
        return x == p[x] ? x : p[x] = find(p[x]);
    }

    void unite(int x, int y) {
        x = find(x), y = find(y);
        if (x == y) return;
        if (sz[x] < sz[y]) swap(x, y);
        p[y] = x;
        sz[x] += sz[y];
    }

    int size(int x) {
        return sz[find(x)];
    }

    bool same(int x, int y) {
        return find(x) == find(y);
    }
};

void solve()
{
    int n, m; cin >> n >> m;
    DSU dsu(n+1);
    vector<tiii> edges;
    rep(i, 1, m+1){
        int a, b; cin >> a >> b;
        if(dsu.same(a, b)){
            edges.pb({a, b, i});
        }
        else dsu.unite(a, b);
    }

    set<int> s;
    rep(i, 1, n+1){
        s.insert(dsu.find(i));
    }
    cout << sz(s) - 1 << endl;

    if(sz(s) == 1) return;

    for (auto tup : edges) {
        auto [u, v, id] = tup;
        int f1 = dsu.find(u);
        if (sz(s) == 1) break;

        if (f1 == *s.begin()) {
            int f2 = dsu.find(*next(s.begin()));
            cout << id << " " << u << " " << *next(s.begin()) << endl;
            dsu.unite(f1, f2);
            s.erase(f1);
            s.erase(f2);
            s.insert(dsu.find(u));
        } else {
            int f2 = dsu.find(*s.begin());
            cout << id << " " << u << " " << *s.begin() << endl;
            dsu.unite(f1, f2);
            s.erase(f1);
            s.erase(f2);
            s.insert(dsu.find(u));
        }
    }

}
\end{lstlisting}